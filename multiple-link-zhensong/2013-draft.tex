\documentclass{article}
\usepackage{url}
%Preamble File for Random Regression Paper

%\usepackage{times}

%Theorems and such

%\usepackage{amsthm}

%\newtheorem{theorem}{Theorem}
%\newtheorem{observation}{Observation}
%\newtheorem{proposition}{Proposition}
%\newtheorem{definition}{Definition}

%\newtheorem{theorem}{Theorem}[section]
%\newtheorem{observation}{Observation}[section]
%\newtheorem{proposition}{Proposition}[section]
%\newtheorem{definition}{Definition}[section]

\usepackage[ruled,vlined]{algorithm2e}
\usepackage{algorithmic}
%\usepackage{amsthm}
\usepackage{amsmath}
\usepackage{amsfonts}
\usepackage{amssymb}
\usepackage{graphicx}
\usepackage{url}
\usepackage{subfigure}
\usepackage{epstopdf}
%\setcounter{MaxMatrixCols}{30}
%\usepackage[ruled,vlined]{algorithm2e}
%\usepackage{algorithmic}
\usepackage{multirow}
\usepackage{subfigure}
\usepackage{ifthen}
\DeclareMathOperator*{\argmax}{argmax}
\DeclareMathOperator*{\argmin}{argmin}
%\DeclareMathOperator{\pattern}{\pi}
\DeclareMathOperator{\Poly}{\mathbf{\mathrm{P}}}
\DeclareMathOperator{\RP}{\mathbf{\mathrm{RP}}}
%\DeclareMathOperator{\FP}{\mathbf{\mathrm{FP}}}
\DeclareMathOperator{\NP}{\mathbf{\mathrm{NP}}}
%\DeclareMathOperator{\E}{\mathbb{E}}

\newcommand{\defterm}{\textbf}

\renewcommand{\d}{\mathbf{d}}

\newcommand{\ZZ}{\mathbf{Z}}

\newcommand{\indep}{\ensuremath{\perp{}\!\!\!\!\!\!\!\perp{}}}
\newcommand{\dep}{\ensuremath{{\perp{}\!\!\!\!\!\!\!\not  \perp{}}}}
%\renewcommand{\L}{\mathcal{L}}

% variables denoting sets of nodes
\newcommand{\BN}{B} % Bayes net
\newcommand{\V}{V} 
\newcommand{\partC}{\mathcal{C}}
\newcommand{\pattern}{\pi}
% for population variables
\newcommand{\A}{\mathbb{A}}
\newcommand{\B}{\mathbb{B}}
\newcommand{\C}{\mathbb{C}}
\newcommand{\U}{\mathbb{U}}
%maybe use P, R as well??
\renewcommand{\P}{P}
\newcommand{\R}{R}
% values of Pop variables = constants
\renewcommand{\a}{a}
\renewcommand{\b}{b}
\renewcommand{\c}{c}

%%%
% for terms = nodes in Functor Bayes Net.
\newcommand{\X}{X}
\newcommand{\Y}{Y}
\newcommand{\Z}{Z}
% Next three are currently only ones eligible for \Mrange and \Prange
\newcommand{\TT}{T}
\newcommand{\TI}{\mathsf{T}}
\newcommand{\UT}{U}
\newcommand{\UI}{\mathsf{U}}
\newcommand{\VT}{V}
\newcommand{\VI}{\mathsf{V}}
\newcommand{\W}{W}
%syntax for values
\newcommand{\z}{z}
\renewcommand{\v}{v}
\newcommand{\x}{x}
\newcommand{\y}{y}
\newcommand{\p}{p}
\newcommand{\s}{s}
% Values tied to terms
\newcommand{\TV}{t}
\newcommand{\TTuple}[1][0.0ex]{\vec{t}\hspace{#1}}
\newcommand{\UV}{u}
\newcommand{\UTuple}[1][0.0ex]{\vec{u}\hspace{#1}}
\newcommand{\VV}{v}
\newcommand{\VTuple}{\vec{v}}
%%%
\newcommand{\weight}{w} % weights
% Formulas
\newcommand{\TF}{\vec{T}}
\newcommand{\UF}{\vec{U}}
\newcommand{\VF}{\vec{V}}
%\newcommand{\TF}{\phi}
%\newcommand{\UF}{\psi}
%\newcommand{\VF}{\omega}
% Database (which is always fully-grounded)
\newcommand{\DB}{\FG{\Delta}}
\newcommand{\QC}{\FG{\Lambda}}
\newcommand{\QCtarget}{\FG{\Lambda_{-\TI}}}
% to define a query conjunction of literals
\newcommand{\Qconj}{\Appendterm{\FG{\TT_{\grounding}} = \TV} {\QC}}

% Annotations marking degree of grounding
\newcommand{\UG}[2][0.0ex]{#2^{-}\hspace{#1}}
\newcommand{\PG}[2][0.0ex]{#2^{\prime}\hspace{#1}}
\newcommand{\FG}[2][0.0ex]{#2^{*}\hspace{#1}}

% Functions returning related terms
\newcommand{\MB}[1]{\mathrm{MB}(#1)}
\newcommand{\Pa}[1]{\mathrm{Pa}(#1)}
\newcommand{\Ch}[1]{\mathrm{Ch}(#1)}

% Grounding
\newcommand{\Ground}[1]{#1_\gamma}
\newcommand{\gndlink}{\backslash}

% Values in ranges of example functors
\newcommand{\Man}{\mathrm{M}}
\newcommand{\Woman}{\mathrm{W}}

% Adding a term to a formula
\newcommand{\sepcup}[1][0.5ex]{\hspace{#1}\cup\hspace{#1}}
\newcommand{\Setaddterm}[2]{#1 \sepcup #2}
\newcommand{\Appendterm}[2]{#1, #2}

% Values in the range of related terms
\newcommand{\Mrange}[1]{\ifthenelse{\equal{#1}{T}}{\TTuple_m}{\ifthenelse{\equal{#1}{U}}{\UTuple_m}{\ifthenelse{\equal{#1}{V}}{\VTuple_m}{\mbox{UNKNOWN
TERM ID}}}}}
\newcommand{\Prange}[1]{\ifthenelse{\equal{#1}{T}}{\vec{t}_{pa}}{\ifthenelse{\equal{#1}{U}}{\vec{u}_{pa}}{\ifthenelse{\equal{#1}{V}}{\vec{v}_{pa}}{\mbox{UNKNOWN
TERM ID}}}}}

\newcommand{\GroundPrange}[1]{\ifthenelse{\equal{#1}{T}}{\vec{t}_{pa,\grounding'}}{\ifthenelse{\equal{#1}{U}}{\vec{u}_{pa,\grounding'}}{\ifthenelse{\equal{#1}{V}}{\vec{v}_{pa,\grounding'}}{\mbox{UNKNOWN
TERM ID}}}}}


% Key functions
\newcommand{\joint}{p}
\newcommand{\jprob}[1]{\theta(#1)}
\newcommand{\cprob}[2]{\theta(#1|#2)}
\newcommand{\estcprob}[3]{\widehat{\theta}(#1|#2;#3)}
%\newcommand{\Gpvar}{\tilde{P}}
\newcommand{\Gpvar}{P}
\newcommand{\Gprob}[2]{\Gpvar(#1 | #2)}
\newcommand{\QFC}{QFC} % query family configuration
\newcommand{\Cvar}{\mathrm{n}}
\newcommand{\Fvar}{\mathrm{p}}
\newcommand{\Count}[2]{\Cvar\left[#1;#2\right]}
\newcommand{\CountC}[3]{\Cvar_{#3}\left[#1;#2\right]}
\newcommand{\Freq}[2]{\Fvar\left[#1;#2\right]}
\newcommand{\Relevant}[1]{#1^{\mathrm{r}}}
\newcommand{\Relcount}[2]{\Relevant{\Cvar}\left[#1;#2\right]}
\newcommand{\RelcountC}[3]{\Relevant{\Cvar_{#3}}\left[#1;#2\right]}
\newcommand{\Relfreq}[2]{\Relevant{\Fvar}\left[#1;#2\right]}
\newcommand{\RelfreqC}[3]{\Relevant{\Fvar_{#3}}\left[#1;#2\right]}
\newcommand{\Range}[1]{\mathrm{Ra}(#1)}
\newcommand{\Vars}[1]{\mathrm{Va}(#1)}
% no longer needed
%\newcommand{\Crossprod}[1]{\mathcal{X}}

%variables for sets of values
\newcommand{\setx}{\set{x}}
\newcommand{\sety}{\set{y}}
\newcommand{\setz}{\set{z}}


%statistics
\newcommand{\score}{S}
\newcommand{\parameters}{\mathit{par}}
\newcommand{\bic}{\mathit{BIC}}
%random variables and graphical models
% number of values in the domain of a random variable
% variables for BNs
\newcommand{\domvals}{k}
\newcommand{\nodevalue}{\x}
\newcommand{\parvalue}{\mathbf{\pi}} % a single assignment of values to a set of 
%parents
\newcommand{\parvals}{l} % number of values of parent state.
\renewcommand{\r}{r} % CP-table row
\newcommand{\nbhd}{{\mathsf {nbdh}}}
\newcommand{\child}{\mathit{child}}
\newcommand{\parent}{\mathit{pa}}
\newcommand{\parents}{\mathbf{pa}}
\newcommand{\Parents}{\mathbf{PA}}
\newcommand{\family}{F} % families, family formulas
\newcommand{\Target}{Y_{\target}}
\newcommand{\MBtarget}{\set{X}_{\target}} % Markov blanket of a target node.
\newcommand{\mbtarget}{\set{x}} % values for the markov blanket of a variable, vector-valued
\newcommand{\mbstates}{m} % number of states in Markov blanket
\newcommand{\vpi}{\mathbf{pa}} % for vectors of variable assignments
\renewcommand{\l}{\ell} % class label
\newcommand{\states}{r} % number of states of a variable
\newcommand{\ssize}{N} % number of rows in join table; size of sample
\newcommand{\frequency}{fr}
\newcommand{\pseudo}{\ast}
\newcommand{\counts}{+}
\newcommand{\weighted}{\ast}
\newcommand{\halpern}{H}
\newcommand{\instance}{I}

%logic notation
\newcommand{\functor}{f}
\newcommand{\fvalue}{v}
\newcommand{\variable}{X} % first-order variable
\newcommand{\population}{\mathcal{P}}
\newcommand{\entity}{x}
\newcommand{\formula}{\phi}
\newcommand{\formulas}{\mathcal{\phi}}
\newcommand{\conjunction}{\set{C}} % 
\newcommand{\outdomain}{V}
\newcommand{\literal}{\ell}
%conjunction of literals
\newcommand{\fterm}{\f} % open function term
\newcommand{\fterms}{F} % set of function terms, also nodes in JBN
\newcommand{\term}{\tau}
\newcommand{\terms}{\bs{\tau}}
\newcommand{\constant}{a}
\newcommand{\constants}{\bs{\constant}}
\newcommand{\gterm}{g} % ground term
\newcommand{\gterms}{\bs{\gterm}} %list of ground terms
\newcommand{\vterm}{x} % variable term
\newcommand{\vterms}{\bs{\vterm}} % list of variable terms
\newcommand{\assign}{A} % assignment of values to Bayes net
\newcommand{\grounds}{\#}
\newcommand{\grounding}{\gamma}
\newcommand{\groundall}{\Gamma}
\newcommand{\vars}{\mathit{Var}} % variables in a conjunction
\newcommand{\igraph}{I} % instance-level dependency graph.
\newcommand{\assignment}{\set{a}}
\newcommand{\atom}{\functor}
\newcommand{\gnode}{\alpha}
\newcommand{\gfamily}{\ground{f}}
\newcommand{\numformulas}{m}
% logic programs
\newcommand{\program}{\mathcal{B}}
\newcommand{\clause}{\mathcal{c}}
\newcommand{\head}{\mathit{head}}
\newcommand{\body}{\mathit{body}}
\newcommand{\crule}{\mathit{cr}} % combining rule
\newcommand{\level}{\mathit{level}} % rank of function symbols in LP

%datbase schema
\newcommand{\rcolumns}{R}
\newcommand{\ecolumns}{E}
\newcommand{\dtable}{T} % can't use \table. Generic database table
\newcommand{\datatable}{D} % generic data table, not necessarily part of database.
\newcommand{\jtable}{J} % join table
\newcommand{\Ejoin}{$J^{+}$}
\newcommand{\jtables}{m}
\newcommand{\rtable}{R} % relationship table
\newcommand{\etable}{E} % entity table.
\newcommand{\ttable}{X} % target table
\newcommand{\nextended}{n}
\newcommand{\row}{r}
\newcommand{\rows}{\mathit{rows}}
\newcommand{\col}{j}
\newcommand{\cols}{\mathit{cols}}
\newcommand{\unary}{\f} % to denote a unary or attribute function
\newcommand{\numatts}{u} % to denote the number of unary or attribute functions.
\newcommand{\g}{g} % alternative for function
\newcommand{\relational}{\mathbf{r}} % denotes a generic relational functors, can be both relationship or descriptive attribute of relationship
\newcommand{\Relation}{R} % denotes a generic boolean relation
% a special type of literal conjunction that assigns a value %to each variable
\newcommand{\class}{c} % the class attribute
\newcommand{\classifier}{\mathcal{C}}
\newcommand{\target}{t} % target object
\newcommand{\feature}{f} % feature or desc attribute of object or link
\newcommand{\features}{\bs{f}} % features 
\newcommand{\attribute}{a} % nonclass attribute of target object
\newcommand{\attributes}{\bs{a}}
\newcommand{\rels}{\bs{R}} % chain of relationships.
\newcommand{\maxpath}{\rho}

%special functions
\newcommand{\AVG}{\it{AVG}}
\newcommand{\instances}{n} % counts number of occurrences in DB
\newcommand{\prob}{p} % frequency of formula true in in DB

%variables denoting graphs or models
\newcommand{\mln}{M}
\newcommand{\G}{G}
\newcommand{\node}{X}
\newcommand{\nodes}{V}
\newcommand{\edges}{E}
\newcommand{\clique}{C}
\newcommand{\cliques}{\mathcal{\clique}}
\newcommand{\cliquevalue}{c}
\newcommand{\graph}{G}
\newcommand{\M}{M}
\newcommand{\J}{J}
\renewcommand{\H}{H}
\newcommand{\K}{K} % component
\renewcommand{\O}{O} % oracle
\renewcommand{\path}{\rho} % path, also foreignkey path
% Markov nets
\newcommand{\potential}{\Psi}
% database schema
\newcommand{\type}{\tau} % to denote a generic type
\newcommand{\E}{E} % for entity tables
\newcommand{\e}{e} % for specific entities
\newcommand{\f}{f}
\newcommand{\new}{\it{new}}
\renewcommand{\c}{c}
\renewcommand{\R}{R} % for relationship tables
%\newcommand{\A}{A} % for attributes
\newcommand{\T}{T} % for tables generically
\newcommand{\New}{N}
\newcommand{\D}{\mathcal{D}} % for database instance
\renewcommand{\S}{\mathcal{S}} % for relational structure as conjunction of literals
\newcommand{\databases}{\set{D}} % the number of databases
\newcommand{\vocab}{\mathcal{\L}} % for logical vocabulary associated with database
\newcommand{\name}{\mathit{name}} % generic attribute
\newcommand{\dom}{\mathit{dom}} % domain of attributes
\newcommand{\etables}{\alpha} % entity tables
\newcommand{\rtables}{\beta} % relationship table number
% specific constructs for examples
\newcommand{\student}{\mathit{Student}}
\newcommand{\I}{\mathit{I}}
\newcommand{\course}{C}
\newcommand{\prof}{\mathit{Professor}}
\newcommand{\person}{\mathit{Person}}
\newcommand{\TA}{\mathit{TA}}
\newcommand{\actor}{\mathit{Actor}}
\newcommand{\age}{\mathit{age}}
\newcommand{\intelligence}{\mathit{intelligence}}
\newcommand{\diff}{\mathit{difficulty}}
\newcommand{\reg}{\mathit{Registered}}
\newcommand{\ra}{\mathit{RA}}
\newcommand{\bt}{\mathit{blood type}}
\newcommand{\grade}{\mathit{grade}}
\newcommand{\gpa}{\mathit{gpa}}
\newcommand{\jack}{\mathit{Jack}}
\newcommand{\jill}{\mathit{Jill}}
\newcommand{\smith}{\mathit{Smith}}
\newcommand{\cmpt}{\mathit{CMPT120}}
\newcommand{\hi}{\mathit{Hi}}
% various constants
\newcommand{\true}{\mathrm{T}}
\newcommand{\false}{\mathrm{F}}
\newcommand{\normalconstant}{Z} % the normalization constant

% orderings
\newcommand{\pred}{\mathit{pred}}
%procedure names and such
\newcommand{\join}{\textsc{Join-Frequencies}}
\newcommand{\linus}{\textsc{Linus }}
\newcommand{\foil}{\textsc{Foil }}
\newcommand{\MLN}{\textsc{MLN}}
\newcommand{\treetilde}{\textsc{TILDE }}

%%%
%undirected models
\newcommand{\pot}{\phi} % potential function
%\newcommand{\theHalgorithm}{\arabic{algorithm}}
\newcommand{\test}{test}
\def\set#1{\mathbf{#1}}
\def\bs#1{\boldsymbol{#1}}
\def\ground#1{\overline{#1}}

\usepackage{proceed2e}
%\usepackage{ltexpprt}
% defines theorem environments etc.

\begin{document}
\title{Outline for a Paper}
\author{Oliver Schulte\\
\\ School of Computing Science\\ Simon Fraser University\\Vancouver-Burnaby, Canada}
\date{\today}
\maketitle

\begin{abstract}
I describe a standard outline for a computer science research paper. The description follows a {\em high-level structure}----what goes where. The goal is to help students get started with writing by providing a skeleton that can be filled in with the details of your research. There are some references to more general guides to scientific writing as well.
\end{abstract}

\section{General Comments}
The goal of this piece is to set up a general format to help get you started on writing a research paper. As a first approach, you can simply go through and fill in the skeleton with your own content. There is an old German saying

\begin{quote}
Students should learn the right forms, apprentices should observe them, masters may change them.
\end{quote}

So eventually you should choose a format that best follows what you are trying to communicate. But this will be many papers down the road; meanwhile, both you and the reader will be happier if you follow standard conventions, even boilerplate.

There are many sources on good scientific writing. Several of these I reread each year. I can provide you with some of the references. I would suggest you do a web search and read at least five promising sources. Here are some.

\begin{itemize}
\item Paul Halmos on writing mathematics. \url{http://www.scribd.com/doc/27073646/Halmos-How-to-Write-Mathematics}
\item David Poole on how to write a research paper \url{http://people.cs.ubc.ca/~poole/HowToWriteResearchPaper.html}
\item Dirk Beyer's miscellaneous tips on writing, dealing with Latex, Bibtex, etc. \url{http://www.sosy-lab.org/~dbeyer/PapersHowToTechnical.html}.
\item Dimitri Bertsekas 10 simple rules for Mathematical Writing. \url{http://ocw.mit.edu/courses/electrical-engineering-and-computer-science/6-253-convex-analysis-and-optimization-spring-2004/projects/ten_rules.pdf}.
\end{itemize}

Some sample quotes. First, from Halmos:

\begin{quote}
The basic problem in writing mathematics is the same as in writing biology, writing a novel, or writing directions for assembling a harpsichord: the problem is to communicate an idea. To do so, and to do it clearly, you must have something to say, and you must have someone to say it to, you must organize what you want to say, and you must arrange it in the order that you want it said in, you must write it, rewrite it, and re-rewrite it several times, and you must be willing to think hard about and work hard on mechanical details such as diction, notation, and punctuation.
\end{quote}

From Poole:

\begin{quote}
    * You should have all of your results ready, and be able to give a "poster overview" of you paper \textbf{two months} before the submission deadline.
    * You should have a complete first draft \textbf{one month} before the submission deadline.

...yes, it takes that long to write a clear paper.
\end{quote}

A good general rule for writing is that good writing is not noticed by the reader. As Hawthorne put it, ``easy reading is damn hard writing''. Having said all that, my experience has been that with conference submissions, the quality of the writing is important, but the general structure and logic of the paper is even more important to the referees. This is because the referees are working on a tight deadline, usually trying to make decisions about 7-10 papers within a week. As a result, they need to quickly get the idea of what you are doing and see if your evidence is convincing. There are often many details that they just miss. 

If you follow all this advice, it will significantly increase the probability of having your paper accepted. However, ultimately the review process is stochastic and noisy, especially with conferences, and so nobody can guarantee acceptance.


\section{Introduction}
The introduction describes the problem that you worked on, and your general approach. Your goal should be to make the reader want to read more of your paper. I approach it with the following mindset: Suppose that you can make the claims you want without having to prove them. In other words, assume that the reader will give you the benefit of the doubt that what you say is true. Then what can you say that will interest them?

\paragraph{Task Description}
What is the problem that you want to solve? 
\paragraph{Motivation} Why is the problem important? To whom does it matter, and why? 
%Example: most enterprises store data in relational format. To exploit their data for decision making and business intelligence, we need to extend statistical machine learning methods to relational data.

\paragraph{Approach} Explain briefly what your general idea is. Especially what is new about it.

\paragraph{Evaluation} Give some idea of how you evaluated your system, and what your results were. With emphasis on the strengths you have been able to demonstrate.

\paragraph{Contributions} Reviewers really appreciate it if you list the top three or so contributions of your paper in point form. Students sometimes think that it should be obvious what the novel contributions are. However, your paper has many details in it, so you can save reviewers from extracting the contributions from it. Also, a reviewer may  not be familar with previous work in your area, or they may just be confused about what you are doing. So by spelling out for them what you have done that's new, you also make sure they don't miss it.

\paragraph{Paper Organization} Explain in outline what each section does, and in what order.

\section{Related Work} Here is where you get into the details of: I did ABCD. There is a previous paper that did ABCD'. It's better to use D instead of D' because... There is also a previous paper that did AB'CD. It's better to use B than B' because...

Related work is very tricky to write because you don't have enough space to cover everything, so you have to select. But if you happen to omit something a referee cares about----like their own work---they will hold it against you, and quite frequently reject on the simple grounds ``the relationship to previous work is unclear''. From their point of view, this way of making a decision also has the advantage that they don't have to study the new details of your own approach, they can simply refer to what they already know. 

On a less cynical note, a good referee tries figure out whether or not what you have done is really new or ``just like x'' where x is already well-known. It's important to try and guess what the top comparison points are, and say something like:

\begin{quote}
Our work is {\em not} like the well-known $x_{1}$ because... Nor is it like the famous $x_{2}$ because... Finally, it is also different from $x_{3}$ because... 
\end{quote}

Once a reviewer is satisfied that there really is something new in your paper, they hopefully will be wiling to make the effort to understand your new ideas.

\section{Background and Notation} 

The tricky part about this section is find the right level for the reviewers. They are very sensitive to this. The worst is to assume that they know something when they don't and you don't explain, or don't explain it enough. But they also hold it against you if they feel that you spend too much time going over the details of what is known in ``the community''---this makes you look like an outsider. For example, at some conferences, you can assume that Bayes net concepts like d-separation are known. At others, you can assume that kernel methods are familar, including things like the primal and dual version of max-margin classifiers. 

\subsection{Notation} {\em Make sure all notation is defined before you use it.} If in doubt, try to use fewer symbols and more English. Halmos has said that ``the best notation is no notation''. In a short conference paper, it's often a good idea to define all your symbols in this section. In a longer paper, it may be a good idea to define symbols close to where you use them.

\subsection{Examples} Readers love examples, working through one to illustrate a definition is fine. For authors, they are a pain, because you already understand your concepts. Plus, it's easy to make small mistakes in an example, especially if you change it around. An ideal to aim for is to have a paper where the reader can read through examples, look through the figures and theorems, and get the idea of your paper without having to read through any details. Another reason to provide an example is this: suppose you guessed wrong about the level of background of  your reviewers, and failed to explain  a background concept so they could understand it. Or suppose you just forgot to define a concept. With an example, the reviewers can often recover from these gaps and get an idea of what you meant.

\section{Your Shiny New Method} This should actually be relatively easy to write, since you are now focusing on your own work. It's often a good idea to start here, like two months before the submission deadline (see Poole above), and leave the trickier early parts for later. This is the part of the paper that is the hardest to read for referees, because it is about details and your new ideas rather than things they already know. So the challenge is to present your system in a way that's as simple and appealing as possible so a referee will be inclined to look at the details.
\subsection{Informal Description, Intuition} Try to say in words what the key idea of your algorithm is.
\subsection{Pseudocode} Make sure all notation is defined.
\subsection{Example} Readers love examples, stepping through a computation is fine.  \subsection{Theoretical Justification} Sometimes this comes first, for instance if your algorithm is based on a mathematical derivation. You could also at this point include a correctness proof, or a complexity analysis (e.g., theoretical upper bound on run-time).
\subsection{Discussion} There may be a need for discussion, for instance why you made certain design choices or how your algorithm is different from that used by others.

\section{Evaluation} This is another important section for referees. Part of the attraction is that they can look at your performance numbers and your methodology and criticize it without having understood the details of your method. Unfortunately, most parts of computer science don't have a fixed evaluation methodology. It is therefore hard to anticipate what reviewers will consider critical. So if in doubt, do more: report more measurements, do more statistical significance tests, try more datasets, try more parameter settings for the methods, look at different sample sizes, etc. I realize that running more experiments can be a lot of work, so students are often tempted to try and do the minimum necessary for acceptance. It's just that nobody knows what that minimum is. Sometimes it's a good idea to send the paper to a workshop, to get some feedback on the evaluation.

Running a set of experiments is likely the most complex programming project you have ever done. It will pay to try and follow a good practice so you don't have to relearn tricks the hard way (i.e., always save your output files). The following website presents some nice advice from someone who has been there: http://arkitus.com/PRML/.

\subsection{Hypotheses} A nice way to structure this section and to engage the referees is to give a list of what you want to establish through your experiments, before you give the details of your experiments. This usually also clarifies things for yourself. Hopefully you have more interesting hypotheses to test than ``I can find some datasets on which the predictive accuracy of my system is higher than that of some previous methods.''

\subsection{Hardware} Describe briefly your set-up, e.g. RAM, Processor, programming language.

\subsection{Datasets} Outline the datasets that you are using. Highlight interesting aspects. Give summary statistics (e.g., data set sizes). Are they real-world or synthetic?

\subsection{Methods Compared} List all of the methods. Make sure you explain which methods are yours and which were proposed by other researchers. Explain settings of methods, and why you chose them.
\subsection{Performance Measures} Explain what you are measuring (e.g., runtime, accuracy) etc. More is better. How are you computing the numbers? E.g., with cross-validation, taking averages? Usually referees like you to add error bars or standard deviations.
\subsection{Results} You need to report results for triples of (method, datasets, metric). This kind of three-dimensional setting is not easy to translate into graphs but you should try. It's best to try and find graphs that support your points; refer back to the hypotheses listed above. I suggest importing  your data into a spreadsheet program, which allows you to produce many charts. Also, there are macros that convert Excel tables into latex tables. This saves you a lot of the pain of producing Latex tables. Avoid color graphs because referees may be printing on black and white, and in any case the conference proceedings won't be in color.
\section{Conclusion} This should be a summary of the main points and what your findings were. Repeat why these results are significant. Often it's a good idea to add some avenues for future work. Often reviewers think ``yes, but you should also have done xyz''. If you can guess what xyz is, you can mention it as an avenue for future work, and then you have shown that you did think of xyz and you acknowledge its importance.
\section{References} Students don't pay nearly enough attention to references. Referees get turned off if the references are incomplete, hard to read, etc., especially if this is is the case for citations to their own work.
\end{document}