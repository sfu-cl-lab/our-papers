\def\year{2021}\relax
%File: formatting-instructions-latex-2021.tex
%release 2021.2
\documentclass[letterpaper]{article} % DO NOT CHANGE THIS
\usepackage{aaai21}  % DO NOT CHANGE THIS
\usepackage{times}  % DO NOT CHANGE THIS
\usepackage{helvet} % DO NOT CHANGE THIS
\usepackage{courier}  % DO NOT CHANGE THIS
\usepackage[hyphens]{url}  % DO NOT CHANGE THIS
\usepackage{graphicx} % DO NOT CHANGE THIS
\urlstyle{rm} % DO NOT CHANGE THIS
\def\UrlFont{\rm}  % DO NOT CHANGE THIS
\usepackage{natbib}  % DO NOT CHANGE THIS AND DO NOT ADD ANY OPTIONS TO IT
\usepackage{caption} % DO NOT CHANGE THIS AND DO NOT ADD ANY OPTIONS TO IT
\usepackage{subfig}
\usepackage{amsmath}
\usepackage{amsthm}
\usepackage{amssymb,amsfonts}
\usepackage[dvipsnames]{xcolor}
\usepackage{multirow}
\usepackage{soul}
\usepackage{lipsum}
\usepackage{booktabs}
\usepackage[switch]{lineno}  %


\frenchspacing  % DO NOT CHANGE THIS
\setlength{\pdfpagewidth}{8.5in}  % DO NOT CHANGE THIS
\setlength{\pdfpageheight}{11in}  % DO NOT CHANGE THIS
%\nocopyright
%PDF Info Is REQUIRED.
% For /Author, add all authors within the parentheses, separated by commas. No accents or commands.
% For /Title, add Title in Mixed Case. No accents or commands. Retain the parentheses.
% \pdfinfo{
% /Title (AAAI Press Formatting Instructions for Authors Using LaTeX -- A Guide)
% /Author (AAAI Press Staff, Pater Patel Schneider, Sunil Issar, J. Scott Penberthy, George Ferguson, Hans Guesgen, Francisco Cruz, Marc Pujol-Gonzalez)
% /TemplateVersion (2021.2)
% } %Leave this
% /Title ()
% Put your actual complete title (no codes, scripts, shortcuts, or LaTeX commands) within the parentheses in mixed case
% Leave the space between \Title and the beginning parenthesis alone
% /Author ()
% Put your actual complete list of authors (no codes, scripts, shortcuts, or LaTeX commands) within the parentheses in mixed case.
% Each author should be only by a comma. If the name contains accents, remove them. If there are any LaTeX commands,
% remove them.

\newcommand{\context}{c}
\newcommand{\expect}{\mathbb{E}}
\newcommand{\expectdiff}{ED}
\newcommand{\scorediff}{\it{SD}}
\newcommand{\latentvariables}{\mathbf{z}}
\newcommand{\inference}{q}
\newcommand{\generation}{p}
\newcommand{\hiddenstate}{\mathbf{h}}
\newcommand{\state}{\mathbf{s}}
\newcommand{\action}{\mathbf{a}}
\newcommand{\reward}{\boldsymbol{r}}
\newcommand{\goal}{g}
\newcommand{\player}{pl}
\newcommand{\pindex}{i}
\newcommand{\prior}{p}
\newcommand{\bin}{\beta}
\newcommand{\boxscore}{b}
\newcommand{\home}{\it{Home}}
\newcommand{\away}{\it{Away}}
\newcommand{\none}{\it{Neither}}
\newcommand{\team}{\it{team}}
\newcommand{\egoal}{\it{goal}}
\newcommand{\Qobs}[2]{Q_{#1}^{\it{obs}}(#2)}
\newcommand{\Qmodel}[3]{\hat{Q}_{#1}(#2,#3)}
\newcommand{\Qbin}[2]{\hat{Q}_{#1}(#2)}
% \newcommand{\observation}{\boldsymbol{x}}
\newcommand{\softmax}{\boldsymbol{\phi}}
\newcommand{\sigmoid}{\boldsymbol{\sigma}}
\newcommand{\observation}{\boldsymbol{o}}
\newcommand{\GaussianParameters}{\boldsymbol{\omega}}
\newcommand{\BernoulliParameters}{\theta}
\newcommand{\system}{VaRLAE\;}

% DISALLOWED PACKAGES
% \usepackage{authblk} -- This package is specifically forbidden
% \usepackage{balance} -- This package is specifically forbidden
% \usepackage{color (if used in text)
% \usepackage{CJK} -- This package is specifically forbidden
% \usepackage{float} -- This package is specifically forbidden
% \usepackage{flushend} -- This package is specifically forbidden
% \usepackage{fontenc} -- This package is specifically forbidden
% \usepackage{fullpage} -- This package is specifically forbidden
% \usepackage{geometry} -- This package is specifically forbidden
% \usepackage{grffile} -- This package is specifically forbidden
% \usepackage{hyperref} -- This package is specifically forbidden
% \usepackage{navigator} -- This package is specifically forbidden
% (or any other package that embeds links such as navigator or hyperref)
% \indentfirst} -- This package is specifically forbidden
% \layout} -- This package is specifically forbidden
% \multicol} -- This package is specifically forbidden
% \nameref} -- This package is specifically forbidden
% \usepackage{savetrees} -- This package is specifically forbidden
% \usepackage{setspace} -- This package is specifically forbidden
% \usepackage{stfloats} -- This package is specifically forbidden
% \usepackage{tabu} -- This package is specifically forbidden
% \usepackage{titlesec} -- This package is specifically forbidden
% \usepackage{tocbibind} -- This package is specifically forbidden
% \usepackage{ulem} -- This package is specifically forbidden
% \usepackage{wrapfig} -- This package is specifically forbidden
% DISALLOWED COMMANDS
% \nocopyright -- Your paper will not be published if you use this command
% \addtolength -- This command may not be used
% \balance -- This command may not be used
% \baselinestretch -- Your paper will not be published if you use this command
% \clearpage -- No page breaks of any kind may be used for the final version of your paper
% \columnsep -- This command may not be used
% \newpage -- No page breaks of any kind may be used for the final version of your paper
% \pagebreak -- No page breaks of any kind may be used for the final version of your paperr
% \pagestyle -- This command may not be used
% \tiny -- This is not an acceptable font size.
% \vspace{- -- No negative value may be used in proximity of a caption, figure, table, section, subsection, subsubsection, or reference
% \vskip{- -- No negative value may be used to alter spacing above or below a caption, figure, table, section, subsection, subsubsection, or reference

\setcounter{secnumdepth}{0} %May be changed to 1 or 2 if section numbers are desired.

% The file aaai21.sty is the style file for AAAI Press
% proceedings, working notes, and technical reports.
%

% Title

% Your title must be in mixed case, not sentence case.
% That means all verbs (including short verbs like be, is, using,and go),
% nouns, adverbs, adjectives should be capitalized, including both words in hyphenated terms, while
% articles, conjunctions, and prepositions are lower case unless they
% directly follow a colon or long dash

\title{Cover Letter for "Learning Agent Representations for Ice Hockey": Improvements upon our Neurips2020 Submission}
\author{}

\begin{document}
\linenumbers  %
\maketitle

\begin{abstract}
This cover letter provides details on our efforts to respond to the comments from Neurips 2020 reviewers and substantially improve the AAAI-2021 submission of the paper “Learning Agent Representations for Ice Hockey” upon our previous submission. We addressed all major comments of the Neurips-2020 reviewers and in fact resolved every major issue raised by them. We provide clarifications and justifications for our major claims according to the comments from \textcolor{Blue}{reviewer 1}, \textcolor{Red}{reviewer 2}, \textcolor{ForestGreen}{reviewer 3} and \textcolor{orange}{reviewer 4}. We mark the comments from reviewer 1, 2, 3 and 4 with \textcolor{Blue}{blue}, \textcolor{Red}{red}, \textcolor{ForestGreen}{green}, and \textcolor{orange}{orange} colors respectively. To compose a concise response letter, we first solve some shared concerns from reviewers and then answer their specific questions separately. 
\end{abstract}

\section{Shared Comments:}
\begin{itemize}
\item \textcolor{Red}{\bf Reviewer 2:}\textcolor{Red}{\it  "I would have liked to have seen comparisons to more fundamental baselines that didn't make the same assumptions, such as other recurrent models and other models meant for multi-agent modelling"}\\
% While the authors clearly demonstrate the strength of their approach compared to its variations (which is appreciated) there are still a limited set of comparisons being made here. I would have liked to have seen comparisons to more fundamental baselines that didn't make the same assumptions, such as other recurrent models and other models meant for multi-agent modelling.
\textcolor{ForestGreen}{\bf Reviewer 3:}\textcolor{ForestGreen}{\it "socialGAN, SoPHie and other multi-agent representation learning approaches should be added as comparison metrics or a reason for not using them should be added as they explicitly learn individual representations with group context."}\\
% socialGAN, SoPHie and other multi-agent representation learning approaches should be added as comparison metrics or a reason for not using them should be added as they explicitly learn individual representations with group context. Contextual information was added into these types of models in prior work (e.g. Tensor fusion) which would serve as a nice comparison for event prediction.
\textcolor{orange}{\bf Reviewer 4:}\textcolor{orange}{"The paper mentions other approaches and it might be
useful to see a comparison to other papers. However, this reviewer acknowledges that the ablations may already
give a general idea of how other papers will do on this dataset. A direct comparison would be preferred under ideal circumstances, however."}\\
In this submission, we include a recently proposed multi-agent sequential data embedding model (named MA-BE). MA-BE implements policy embedding for multiple agents, which is an ideal comparison method for our approach.
SocialGAN~\cite{Gupta18SocialGan} and SoPHie~\cite{Sadeghian2019SoPhie} are designed for predicting agents' trajectory without explicitly learning player representations, and therefore they are not directly applicable to embed player identities. We provide a short discussion of both models and other multi-agent modeling methods in our related work section by analyzing both their strength and their limitations. %implements a discriminator to judge whether the generated trajectory is real, but we
% are predicting acting player and it is less rational to judge whether the player id is real. 

% We mentioned modelling player interactions for play-by-play data as a topic for future work in our conclusion.

\item \textcolor{ForestGreen}{\bf Reviewer 3:}\textcolor{ForestGreen}{\it “Can the authors provide some key insights from the proposed approach that was missing in this and other prior work on shot prediction."}\\
% The shot quality prediction is similar to the results reported in "“Quality vs Quantity”: Improved Shot Prediction in Soccer using Strategic Features from Spatio temporal Data". Can the authors provide some key insights from the proposed approach that was missing in this and other prior work on shot prediction.
\textcolor{orange}{\bf Reviewer 4:}\textcolor{orange}{\it "The expected goal results are not conclusive. It is unclear that the ladder aspect of the architecture is providing an improvement on this application."}\\
% empirical evaluation: There are number of weaknesses with regard to the empirical evaluation. Most directly, the expected goal results (fig 2) are not conclusive. It is unclear that the ladder aspect of the architecture is providing an improvement on this application task.
In this submission, we provide a more detailed discussion for the results of our expected goal experiment and explain the leading performance achieved by our \system.  
Prior works on ice hockey shot prediction commonly do not take into account the identity of the shooter, and thus the prediction model fails to recognize the difference between the top and below-average shooters.
From our domain knowledge, the scoring chance is higher for top shooters v.s. average shooters under similar game context, so modeling shooter-specific effects (by learning player representation) can significantly improve the accuracy of expected goal estimation. 
\system applies our player representation framework and learns the most distinguishable player representation among all baseline models (see the player identification results in table 1). It explains the advantage of our \system.
%A player's preference (e.g., left/right hand) can also influence his performance, so the representation should be able to identify players under different game context, where \system performs better (see Table 1). 
%The leading performance is because 
\item \textcolor{Red}{\bf Reviewer 2:}\textcolor{Red}{\it "I was somewhat disappointed by the broader impacts section. It seems to me that a model like this is likely only usable by teams with substantial technical resources or the ability to acquire those resources. "}\\
\textcolor{orange}{\bf Reviewer 4:}\textcolor{orange}{Broader impacts section fails to point out the main application of this work is world state
estimation with an emphasis on individual person tracking when dealing with a large number of known individuals.}\\
% The authors focus almost entirely on positive outcomes. It seems to me that a model like this is likely only usable by teams with substantial technical resources or the ability to acquire those resources
%  As such, it may lead to an increased inequality between the top and bottom teams. In addition, given that models like this can only draw inferences from within a learned distribution there's little room for players to grow or change, meaning that a model like this may also increase inequality between players.
We revise our broader impacts section according to the suggestions from reviewers. To reduce inequality, we will release our code, to help level the analytics playing field. While technical skills do require resources, professional scouts are even more expensive. We believe our mode can help low-ranked team that can not afford professional analyst.
% The computing resource is relatively cheaper than hiring professional analytics. 
Our model focuses only on a player's professional skills without considering race, gender, or age, which encourages fairness and reduces bias. We can extend the model to capture player development over time, as it is suggested by the reviewers.

\end{itemize}

\section{\bf Comments from \textcolor{Blue}{Reviewer 1}}

\begin{itemize}
    \item \textcolor{Blue}{\it "The hierarchical latent variables are what distinguishes VaRLAE from other baselines presented in the paper. I would have liked to see some analysis of all the latent variables, not just ones at the lowest level."}\\
    Our new submission adds visualizations for the latent variables on the higher layers (of our ladder structure) as complement results for our embedding visualization experiment.
    The latent variables at higher levels, for example $\latentvariables_{\state,t}$, have no access to $\reward_{t}$ or $\action_{t}$ (This is where our contextualized model differs from the traditional ladder structure). Compared to our player representation ($\latentvariables_{\reward,t}$ which conditions on $\state_{t},\action_{t},\reward_{t}$ and contains the most complete information about each player.), they are less informative. Specifically, latent values from the higher levels distinguish players less, and show a smaller shrinkage effect:  many points are distributed around the mean. Similar results were observed in previous works~\cite{SonderbyLadderVAE16}, and we provide a more detailed discussion about this phenomenon in our paper.
    
    \item \textcolor{Blue}{\it "The main takeaway for the embedding visualization is also unclear. The embedding visualization looks like it’s mainly highlighting the difference between variational and non-variational models. How do the embeddings compare with those from CVRNN, the best baseline?"}\\
    Our main contribution is the idea of Player representation through Player Generation. CVRNN and \system are different architectures for implementing this fundamental idea. Since both methods use the same general idea, we expected their visualization to look similar. In particular, both exhibit a shrinkage effect leading to similar T-SNE projections. 
    The key point of our embedding visualization is to show the difference of a model without a shrinkage effect, namely traditional auto-encoder (CAERNN). To clarify our claim, we have emphasized this point in our new submission.
    % is to illustrate the shrinkage effect, so it is rational to compare a shrinkage encoder (\system) and a , which have the {\it same} format of input and output but a different encoder design.
    \item\textcolor{Blue}{\it "The experimental results on the downstream tasks show marginal improvements over the best baseline. The performance using VaRLAE player representations is on par with CVRNN player representations for both expected goal estimation and score difference prediction."}\\
    % The experimental results on the downstream tasks show marginal improvements over the best baseline. The performance using VaRLAE player representations is on par with CVRNN player representations for both expected goal estimation (comparable F2-score and AUC) and score difference prediction (lower mean, but larger variance within range). The effectiveness of the learned representations is unclear some more experiments (or domains)
    Our paper covered three popular tasks in the Ice hockey domain.
    CVRNN is indeed the strongest 
    %baseline according to our 
    ablation method implementing our Representation-Through-Generation framework. Our \system  beats it by an average of 8\% (over 12\% for players with sparse participation) in player identification. Expected goals results are mixed: CVRNN has higher precision, \system has the second-best precision, and achieves overall best performance (Recall and F1-score). \system has a minimum error in the score difference prediction experiment. These results show the promising performance of \system.
    
\end{itemize}


\section{Comments from \textcolor{red}{Reviewer 2}}
\begin{itemize}
    \item \textcolor{Red}{\it "The paper is very dense and at times lacking in clarity. In particular, the end of the introduction essentially walks the reader through the whole approach, but leaves out a number of important details. The paper is well-written at a local level. However, I found that the overall structure was somewhat confusing."}\\
    We revise our AAAI submission according to the comments from the reviewers. Our introduction briefly describes our model designs and discusses the main advantage of applying our \system model. The following sections ( including the "Player Representation Framework" and "Variational Agent Encoder") clarify and elaborates on the points mentioned in our introduction.
\end{itemize}
% The paper is very dense and at times lacking in clarity. In particular, the end of the introduction essentially walks the reader through the whole approach, but leaves out a number of important details. This was confusing to me, as a reader, as it was unclear to me when or where I might find these missing details. In addition, a large portion of the content of the back half of the introduction is repeated in the later sections, which further adds to this confusion. 

\section{Comments from \textcolor{orange}{Reviewer 4}}
\begin{itemize}
    \item \textcolor{orange}{"There is reason to believe that the VaRLAE architecture is applicable to more domains than just hockey. It would strengthen the paper to see this applied to more sports datasets or even non-sports datasets that share similarities with respect to individual tracking (eSports for example?) with a large number of entities."}. \\
    % There is reason to believe that the VaRLAE architecture is applicable to more domains than just hockey. It would strengthen the paper to see this applied to more sports datasets or even non-sports datasets that share similarities with respect to individual tracking (eSports for example?) with a large number of entities. The hockey dataset is fairly unique, however, with features (e.g. puck data), and this type of experimentation will tell readers how much the particular architecture is tuned to the particularities of this one dataset.
    Our \system can indeed be applied to other team sports, as we mention in our conclusion, but the goal of this paper is to provide a thorough in-depth evaluation of several tasks in the domain of ice hockey. 
    % However, play-by-play data is not readily available for other sports (???). 
    % the sports dataset is confidential and we have only match records for Ice Hockey (examples will be published for experiments).  but this is beyond our scope.
\end{itemize}

\bibliography{master}
\end{document}